\documentclass[11pt]{book}
\parskip0pt
\usepackage{polski}
\usepackage[utf8]{inputenc}
\usepackage{listing}
\usepackage{enumerate}
\usepackage{amsmath}
\usepackage{amssymb}
\usepackage[colorlinks=true]{hyperref}

\author{Mariusz Różycki, Cezary Obczyński}
\title{Informatyka matematyczna. Semestr I}

\begin{document}
\frontmatter
\maketitle

\newpage
~\vfill
\thispagestyle{empty}
\noindent Copyright \copyright 2014 Mariusz Różycki, Cezary Obczyński\\

\noindent Licensed under the Apache License, Version 2.0 (the "License");
you may not use this file except in compliance with the License.
\vspace{1em}

\noindent You may obtain copy of the License at 
\url{http://www.apache.org/licenses/LICENSE-2.0.html}.
\vspace{1em}

\noindent Unless required by applicable law or agreed to in writing, software
distributed under the License is distributed on an "AS IS" BASIS,
WITHOUT WARRANTIES OR CONDITIONS OF ANY KIND, either express or implied.
See the License for the specific language governing permissions and
limitations under the License.

\tableofcontents

\mainmatter

\part{Wprowadzenie}
\chapter{Sprawy techniczne}
\section{O tym podręczniku}
Podręcznik ten jest uzupełnieniem do zajęć informatyki matematycznej w klasach
matematyczno-fizycznych w Liceum Ogólnokształcącym im. Marsz. Stanisława
Małachowskiego w Płocku.

Zawiera omówienia zagadnień, które poruszane są na lekcjach. Często
prezentować może odmienne podejście do tematu, dostarczając dodatkowy
materiał do nauki dla osób, które mają problemy ze zrozumieniem materiału
w czasie zajęć.

Podręcznik ten zawiera również dodatkowy materiał dla osób zainteresowanych
tematem, które chcą poszerzać swoją wiedzę ponad zakres tematów omawianych
na lekcjach.

Jako że tematem tego podręcznika jest wprowadzenie do programowania,
znajduje się w nim również wiele ćwiczeń praktycznych. Część z nich zostanie
omówiona na zajęciach w szkole, reszta pozostaje dla uczniów do samodzielnego
rozwiązania w domu. Wykonywanie ich jest wysoce zalecane, jako że ułatwiają
one przyswojenie materiału, który stanowi podstawę nauki informatyki.

\section{Licencje i prawa autorskie}
Poniższy podręcznik udostępniony jest na licencji Apache w wersji 2.0.
Pełny tekst tej licencji (w języku angielskim) znaleźć można w internecie
pod adresem \url{http://www.apache.org/licenses/LICENSE-2.0.html}.

\section{Korzystanie z pracowni komputerowej}
Sala informatyczna, z której będziemy korzystać przez najbliższe dwa lata
różni się od sal informatycznych, z których korzystaliście do tej pory.
Dla wygody zarówno waszej jak i prowadzących zajęcia, warto wiedzieć czym
dokładnie wyróżnia się ta pracownia oraz w jaki sposób prawidłowo z niej
korzystać.

Poniższe zasady odnoszą się również do dodatkowych zajęć z informatyki, które
odbywać będą się przy użyciu tego samego sprzętu.
% TODO (24/07/14): napisać resztę sekcji

\section{Zakres materiału informatyki w pierwszym semestrze}
W pierwszym semestrze skupimy się na nauce programowania od podstaw. Najpierw 
przyjrzymy się podstawowym zagadnieniom w programowaniu imperatywnym (nie 
przejmuj się, jeżeli nie wiesz co to oznacza) w oparciu o C++. Omówimy motywy
pojawiające się niemal w każdym współczesnym języku programowania: zmienne,
wyrażenia warunkowe, pętle, funkcje.

Następnie skupimy się na programowaniu obiektowym w języku Java. Przyjrzymy
się pojęciom obiektu i klasy. Poznamy zagadnienia definiujące paradygmat
programowania obiektowego, czyli enkapsulację, dziedziczenie i polimorfizm.

\chapter{Czym zajmuje się informatyka?}
\section{Zera i jedynki}
\section{,,Duże'' liczby, teksty, obrazy}
\section{Teoria i praktyka}


\part{Podstawy programowania}
\chapter{Wprowadzenie}
\section{Na czym polega programowanie?}
\section{Czym jest kompilator?}
\section{Korzystanie ze środowiska Code::Blocks}
\section{Pierwszy program}
\section{Wypisywanie danych}
\section{Przeprowadzanie prostych obliczeń}

\chapter{Zmienne i stałe}
\section{Tworzenie zmiennych}
\section{Modyfikowanie wartości zmiennych}
\section{Podstawowe typy danych}
\section{Wykorzystywanie zmiennych w obliczeniach}
\section{Pobieranie danych od użytkownika}

\chapter{Wyrażenia warunkowe}
\section{Porównywanie wartości}
\section{Wyrażenie \texttt{if}}
\section{Wyrażenie \texttt{else}}

\chapter{Pętle}
\section{Pętla \texttt{while}}
\section{Pętla \texttt{for}}

\chapter{Funkcje}
\section{Abstrakcja}
\section{Duplikacja kodu}
\section{Tworzenie funkcji bez argumentów i wartości zwracanej}
\section{Funkcje przyjmujące argumenty}
\section{Zwracanie wartości}
\section{Rekurencja}


\part{Podstawy programowania obiektowego}
\chapter{Wprowadzenie}
\section{Korzystanie ze środowiska IntelliJ IDEA}
\section{Pierwszy program w języku Java}
\section{Podobieństwa i różnice między językami Java i C++}

\chapter{Obiekty i klasy}
\section{Abstrakcja: przypomnienie}
\section{Obiekt jako przykład abstrakcji}
\section{Klasa jako szablon dla obiektów}
\section{Definiowanie własnych klas}

\chapter{Enkapsulacja}
\section{Czym jest enkapsulacja?}
\section{Poziom dostępu \texttt{public}}
\section{Poziom dostępu \texttt{private}}

\chapter{Dziedziczenie i polimorfizm}
\section{Duplikacja kodu: przypomnienie}
\section{Dziedziczenie}
\section{Dziedziczenie w języku Java}
\section{Polimorfizm}

\chapter{* Interfejsy}
\section{Klasa abstrakcyjna}
\section{Wielokrotne dziedziczenie}
\section{Interfejsy}
\section{Biblioteka standardowa Javy}

\end{document}
