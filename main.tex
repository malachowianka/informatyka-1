\documentclass[11pt]{book}
\parskip0pt
\usepackage{polski}
\usepackage[utf8]{inputenc}
\usepackage{listings}
\usepackage{enumerate}
\usepackage{amsmath}
\usepackage{amssymb}
\usepackage[colorlinks=true]{hyperref}
\usepackage{epigraph}
\usepackage{etoolbox}
\usepackage{framed}
\usepackage{color}

\setlength\epigraphwidth{0.8\textwidth}
\setlength\epigraphrule{0pt}
\makeatletter
\patchcmd{\epigraph}{\@epitext{#1}}{\itshape\@epitext{#1}}{}{}
\makeatother

\setcounter{tocdepth}{1}

\lstset{basicstyle=\ttfamily,
        language=C++,
        captionpos=b,
        numbers=left,
        numbersep=15pt,
        numberstyle=\footnotesize\ttfamily,
        xleftmargin=30pt,
        aboveskip=2em,
        belowskip=2em,
        frame=trBL,
        framexleftmargin=25pt,
        inputencoding=utf8,
        showstringspaces=false,
        escapeinside={(@}{@)}}

\definecolor{shadecolor}{gray}{0.95}

\newenvironment{try}
{ \vspace{1em} \noindent
  \definecolor{shadecolor}{gray}{0.7}
  \begin{minipage}{\textwidth} 
  \begin{shaded} 
  \textbf{Spróbuj!} }
{ \end{shaded} 
  \end{minipage} 
  \vspace{1em} }

\newenvironment{notice}
{ \vspace{1em} \noindent
  \definecolor{shadecolor}{gray}{0.9}
  \begin{minipage}{\textwidth} 
  \begin{shaded} \textbf{Uwaga!} }
{ \end{shaded} 
  \end{minipage} 
  \vspace{1em} }

\author{Mariusz Różycki, Cezary Obczyński}
\title{Informatyka matematyczna. Semestr I}

\begin{document}
\frontmatter
\maketitle

\newpage
~\vfill
\thispagestyle{empty}
\noindent Copyright \copyright 2014 Mariusz Różycki, Cezary Obczyński\\

\noindent Licensed under the Apache License, Version 2.0 (the "License");
you may not use this file except in compliance with the License.
\vspace{1em}

\noindent You may obtain copy of the License at 
\url{http://www.apache.org/licenses/LICENSE-2.0.html}.
\vspace{1em}

\noindent Unless required by applicable law or agreed to in writing, software
distributed under the License is distributed on an "AS IS" BASIS,
WITHOUT WARRANTIES OR CONDITIONS OF ANY KIND, either express or implied.
See the License for the specific language governing permissions and
limitations under the License.

\tableofcontents

\mainmatter

\part{Wprowadzenie}
\chapter{Sprawy techniczne}
\section{O tym podręczniku}
Podręcznik ten jest uzupełnieniem do zajęć informatyki matematycznej w klasach
matematyczno-fizycznych w Liceum Ogólnokształcącym im. Marsz. Stanisława
Małachowskiego w Płocku.

Zawiera omówienia zagadnień, które poruszane są na lekcjach. Często
prezentować może odmienne podejście do tematu, dostarczając dodatkowy
materiał do nauki dla osób, które mają problemy ze zrozumieniem materiału
w czasie zajęć.

Podręcznik ten zawiera również dodatkowy materiał dla osób zainteresowanych
tematem, które chcą poszerzać swoją wiedzę ponad zakres tematów omawianych
na lekcjach.

Jako że tematem tego podręcznika jest wprowadzenie do programowania,
znajduje się w nim również wiele ćwiczeń praktycznych. Część z nich zostanie
omówiona na zajęciach w szkole, reszta pozostaje dla uczniów do samodzielnego
rozwiązania w domu. Wykonywanie ich jest wysoce zalecane, jako że ułatwiają
one przyswojenie materiału, który stanowi podstawę nauki informatyki.

\section{Licencje i prawa autorskie}
Poniższy podręcznik udostępniony jest na licencji Apache w wersji 2.0.
Pełny tekst tej licencji (w języku angielskim) znaleźć można w internecie
pod adresem \url{http://www.apache.org/licenses/LICENSE-2.0.html}.

\section{Korzystanie z pracowni komputerowej}
Sala informatyczna, z której będziemy korzystać przez najbliższe dwa lata
różni się od sal informatycznych, z których korzystaliście do tej pory.
Dla wygody zarówno waszej jak i prowadzących zajęcia, warto wiedzieć czym
dokładnie wyróżnia się ta pracownia oraz w jaki sposób prawidłowo z niej
korzystać.

Poniższe zasady odnoszą się również do dodatkowych zajęć z informatyki, które
odbywać będą się przy użyciu tego samego sprzętu.
% TODO (24/07/14): napisać resztę sekcji

\section{Zakres materiału informatyki w pierwszym semestrze}
W pierwszym semestrze skupimy się na nauce programowania od podstaw. Najpierw 
przyjrzymy się podstawowym zagadnieniom w programowaniu imperatywnym (nie 
przejmuj się, jeżeli nie wiesz co to oznacza) w oparciu o C++. Omówimy motywy
pojawiające się niemal w każdym współczesnym języku programowania: zmienne,
wyrażenia warunkowe, pętle, funkcje.

Następnie skupimy się na programowaniu obiektowym w języku Java. Przyjrzymy
się pojęciom obiektu i klasy. Poznamy zagadnienia definiujące paradygmat
programowania obiektowego, czyli enkapsulację, dziedziczenie i polimorfizm.

\chapter{Czym zajmuje się informatyka?}
\epigraph%
{Komputer z czarną magią, oczywiście, nie ma nic wspólnego. Przeciwnie, jej
istota jest w zasadzie dziecinnie prosta. Prosta w sensie idei, a nie
technicznej konstrukcji. W każdym razie prosta na tyle, że nie potrzeba z góry
zakładać, że się tego nigdy nie zrozumie.}%
{--- \textup{Krzysztof Zanussi}, Komputery}

Powyższy cytat, pochodzący z powstałego w 1967 roku filmu dokumentalnego
,,Komputery'', pomimo upływu lat i nieustannego rozwoju technologii, pozostaje
wciąż prawdziwy i aktualny. Dzisiaj komputery traktujemy bardziej jako
czarne skrzynki, które po prostu działają, nie zastanawiając się nad tym
co konkretnie dzieje się w środku.

Nie jest to co prawda wiedza niezbędna na co dzień. Jednakże w czasach, kiedy
niemal każdy nosi w kieszeni miniaturowy komputer, na pewno warto jest wiedzieć,
chociaż w uproszczeniu, jak te urządzenia działają w środku.

\section{Zera i jedynki}
Pierwszą rzeczą, którą należy sobie uświadomić, jest to, że komputery w gruncie
rzeczy są dość głupie. Potrafią operować jedynie na dwóch wartościach: 0 i 1.
Wszystko sprowadza się do długich ciągów tych tylko dwóch znaków: każdy film,
artykuł, zdjęcie, program, strona internetowa, system operacyjny. Same zera
i jedynki. Pojedynczą taką wartość nazywamy \textit{bitem}.

Ważne jest również to, że każde zero jest takie samo jak każde inne zero, a
każda jedynka jest identyczna każdej innej jedynce. Zero będące częścią twojego
ulubionego filmu nie różni się niczym od zera znajdującego się w reprezentacji
wiadomości wysłanej przez Facebooka, przelewu bankowego czy arcydzieła
narysowanego w Paincie przez twojego trzyletniego kuzyna. Wszystkie zera są
sobie równe i wszystkie jedynki są sobie równe, a do tego są jedynymi
dostępnymi wartościami.

Komputer nie potrafi też zbyt wiele z tymi bitami zrobić. Oprócz ich
przechowania, może wykonywać na nich tylko najprostsze operacje logiczne:
,,i'', ,,lub'', ,,nie'' oraz ,,albo''.

Skoro więc komputer na najniższym poziomie potrafi tak niewiele, jakim cudem
możemy przy ich pomocy wykonywać skomplikowane obliczenia, słuchać muzyki,
czy w czasie rzeczywistym porozumiewać się z ludźmi znajdującymi się na
drugiej stronie globu?

Cały sekret kryje się w skali. Pojedynczy bit niesie ze sobą niewiele
informacji, nie daje nam też zbyt wiele możliwości. Jednakże liczba bitów,
jakie komputer potrafi przechować liczy się w bilionach. Liczba prostych
operacji, które komputer może wykonać w ciągu sekundy liczy się w dziesiątkach
(jeżeli nie setkach) miliardów.

\section{Zapisywanie liczb, tekstu, zdjęć i filmów}
Zastanówmy się teraz w jaki sposób możemy użyć ciągów zer i jedynek, żeby
zapisywać liczby oraz tekst.

\subsection{Systemy liczbowe}
Przyjrzyjmy się temu w jaki sposób zapisujemy liczby na co dzień. Przecież,
jak by nie patrzeć, również używamy do tego ciągów znaków. Główną różnicą
jest to, że do dyspozycji mamy 10 różnych znaków, a nie 2.

Spójrzmy na liczbę 1729 i zastanówmy się co dokładnie oznacza każda z cyfr
w tym zapisie. Mogliście się spotkać z określeniami takimi jak
,,cyfra jedności'', ,,cyfra dziesiątek'' itp. Idąc od prawej strony, każda
kolejna cyfra w zapisie dziesiętnym oznacza ile razy dana potęga liczby 10
występuje w danej liczbie.

I tak 9 jest cyfrą jedności, zatem określa ile jedności znajduje się w liczbie
1729. Dalej 2 jest cyfrą dziesiątek, 7 jest cyfrą setek, a 1 cyfrą tysięcy.

Stąd 1729 możemy inaczej zapisać jako $1\times 1000 + 7\times 100 + 2\times 10
+ 9\times 1$ lub, zapisując potęgi dziesiątki w postaci wykładniczej,
$1\times 10^3 + 7\times 10^2 + 2\times 10^1 + 9\times 10^0$.

Do zapisu liczb w systemie dziesiętnym wystarczy 10 różnych symboli. Nie
potrzebujemy osobnych znaków do przedstawiania większych wartości, ponieważ
w zamian używamy ponownie tych samych symboli na kolejnych miejscach.
Dziesięć jedności to jedna dziesiątka, dziesięć dziesiątek to jedna setka itd.

Co gdyby zamienić liczbę 10 jakąś inną? Na przykład 8? Wtedy każda kolejna
cyfra w zapisie oznaczałaby kolejną potęgę 8, stąd 123 w systemie dziesiętnym
to $1 \times 8^2 + 2 \times 8^1 + 3 \times 8^0 = 1 \times 64 + 2 \times 8
+ 3 \times 1 = 64 + 16 + 3 = 83$ w systemie dziesiętnym.

Zwróć też uwagę, że do zapisu dowolnej liczby naturalnej wystarczą nam tylko
cyfry od 0 do 7, ponieważ 8 jedności możemy zapisać jako jedną ósemkę,
8 ósemek jako jedną 64.

My jednak mamy do dyspozycji tylko 2 symbole: 0 i 1. W takim razie zamiast
10 czy 8 powinniśmy jako podstawy zapisu liczb użyć liczby 2. W takim
zapisie każda kolejna cyfra oznaczałaby kolejną potęgę dwójki. I tak 1001
w systemie dwójkowym to $1\times 2^3 + 0\times 2^2 + 0\times 2^1 + 1\times 1
= 1 \times 8 + 1 \times 1 = 8+1 = 9$.

System dwójkowy, zwany inaczej binarnym, używany jest do przedstawiania w
pamięci komputera nieujemnych liczb całkowitych.

Zwróć uwagę, że przy użyciu 3 bitów możemy zapisać liczby od 0 do 7, czyli
od 0 do $2^3-1$. Bardziej ogólnie, przy użyciu $n$ bitów możemy w ten sposób
zapisać liczby od 0 do $2^n-1$ włącznie.

\subsection{ASCII}
W latach 60. XIX wieku organizacja American Standards Association opracowała
system kodowania liter alfabetu angielskiego (osobno małych i wielkich), cyfr, 
znaków interpunkcyjnych, niektórych operatorów matematycznych oraz pewnych 
znaków specjalnych, przy użyciu liczb. W sumie 128 różnych znaków, co wymaga
7 bitów do zapisu każdego z nich. Zostało to jednak zaokrąglone do 8 bitów,
jako że większość komputerów operuje na 8-bitowych \textit{bajtach}.

W oryginalnym kodzie ASCII ósmy bit pozostaje nieużywany, jednak powstało
wiele rozszerzeń, które wykorzystują ten bit, a dodatkowe 128 dostępnych dzięki
temu liczb wykorzystuje do zapisu narodowych znaków diakrytycznych. W przypadku
niektórych języków (np. chiński, japoński, koreański) te 128 dodatkowych liczb
nie wystarcza, stąd powstały bardziej zaawansowane kodowania, jak Unicode,
które wykorzystują więcej niż jeden bajt do zapisu jednego znaku.

Powróćmy jednak na chwilę do ASCII. Wielka litera ,,A'' w tym kodzie zapisywana
jest jako liczba 65, a każdej kolejnej literze angielskiego alfabetu
przypisana jest kolejna liczba, aż do liczby 91, która reprezentuje literę
,,Z''. Stąd słowo ,,INFORMATYKA'' zapisane zostanie jako ciąg: 73, 78, 70, 79,
82, 77, 65, 84, 89, 75, 65. 

\subsection{Zdjęcia i filmy}
Najprostszym sposobem zapisu zdjęcia jest podzielenie go na małe kwadraciki,
zwane pikselami. Każdemu pikselowi przypisujemy trzy wartości, oznaczające
kolejno intensywność kolorów czerwonego, zielonego i niebieskiego (RGB). 
Najczęściej używa się 8 bitów (a zatem 256 różnych wartości) dla każdego 
z tych kolorów, dając w sumie 24 bity na piksel (stąd 24-bitowe grafiki).

Taki sposób zapisywania obrazów nazywamy \textit{mapą bitową}. Jest to bardzo
mało wydajny system. Zdjęcie w rodzielczości FullHD (1920x1080 pikseli)
zapisane w tym systemie zajmie $1920 \times 1080 \times 24$, czyli 49~766~400
bitów. Jednakże do określenia rozmiaru plików z reguły używa się nieco innych
jednostek: 8 bitów (8b) to jeden bajt (1B), 1024 bajty to 1 kilobajt (1kB),
1024 kilobajty to 1 megabajt (1MB), a 1024 megabajty to 1 gigabajt (1GB).

Zatem 50 milionów bitów to w przybliżeniu 12 milionów bajtów, czyli około 12MB.
Być może zdajesz sobie sprawę, że w rzeczywistości zdjęcie w tej rozdzielczości
zajmie dużo mniej miejsca, z reguły około 400-500kB. To dzięki kompresji, która
jednak wykracza poza zakres materiału tego kursu.

Stąd do zapisu filmu już krótka droga. Wystarczy zapisać osobno każdą klatkę
filmu. Do tego, oczywiście, trzeba zastosować sporo kompresji, ponieważ bez
niej jedna sekunda filmu FullHD (24 klatki) zajmowałaby prawie 300MB.

\section{Nauka o informacji}
Informatyka to jednak dużo więcej niż tylko zapisywanie filmów przy użyciu
zer i jedynek. Informatyka, jako nauka, zajmuje się \textit{informacją}.
To znaczy, między innymi, jak przechowywać informacje, jak je skutecznie
przetwarzać, w jaki sposób szybko wykonywać obliczenia, jak upewnić się, że
ich wynik jest poprawny oraz jak zapewnić danym bezpieczeństwo. Informatyka 
zajmuje się też decydowaniem co w ogóle można obliczyć, ile czasu oraz pamięci 
to zajmie.

My skupimy się na praktycznych zastosowaniach informatyki.

\part{Podstawy programowania}
\chapter{Wprowadzenie}
\epigraph%
{Informatyka jest tak samo nauką o komputerach, jak astronomia jest nauką
o teleskopach.}%
{--- autor nieznany}

Informatyka, jako nauka, nie ogranicza się tylko do komputerów czy
programowania, tak samo jak astronomia nie ogranicza się do patrzenia w gwiazdy.
Jednak tak samo jak astronomia bez teleskopów, informatyka bez komputerów
ogranicza się do nudnego wkuwania teorii. Nie zmienia to jednak faktu, że
naprawa drukarki czy konfiguracja sieci domowej ma się do informatyki tak,
jak stosowanie środków czyszczących do chemii.

\section{Na czym polega programowanie?}
Komputer jest głównym narzędziem informatyka. Z komputerem się trzeba jednak
w jakiś sposób porozumieć. Problem polega na tym, że ludzkich języków komputer
nie rozumie, a język, którym posługuje się komputer jest raczej zbyt
skomplikowany dla ludzi, aby był praktyczny.

Stąd powstało wiele języków pośrednich, które człowiek może zrozumieć, a do
tego łatwo jest je przetłumaczyć na język zrozumiały dla komputera.

Języki te nazywamy językami programowania. Przy ich użyciu możemy redagować
ciągi instrukcji (zwane programami), które komputer następnie wykona. Trzeba 
jednak pamiętać, że komputer nie potrafi się domyślić co programista miał na 
myśli. Stąd programy powinny być jednoznaczne.

\section{Czym jest kompilator?}
\section{Korzystanie ze środowiska Code::Blocks}
\section{Pierwszy program}
Utwórz nowy, pusty plik i skopiuj do edytora poniższy fragment kodu:

\begin{lstlisting}[caption={Hello, world!}]
(@ \label{lst:hello} @)
#include <iostream>
using namespace std;

int main() {
  cout << "Hello, world!" << endl;

  return 0;
}
\end{lstlisting}

Zapisz go jako, na przykład, \texttt{hello.cpp}, skompiluj i uruchom.

\begin{notice}
Nazwa pliku z kodem źródłowym nie ma większego znaczenia dla kompilatora
(poza nielicznymi wyjątkami, o których niżej). Nazywaj je według własnego
uznania, tak, aby łatwo byłoby Ci odnaleźć właściwy plik, kiedy będziesz go
potrzebować.

C++ w podstawowej formie koduje tekst przy użyciu ASCII, więc polskie znaki
mogą sprawiać problemy. Staraj się ich unikać, zarówno w nazwach plików z kodem 
źródłowym, jak i w samym kodzie.

W nazwach plików unikaj także spacji. Jeżeli musisz, użyj zamiast niej
podkreślenia (czyli ,,\_'') lub stosuj tzw. \textit{camelCase}, czyli
każde słowo (oprócz pierwszego) pisz wielką literą, całkowicie pomijając spacje.
\end{notice}

Powyższy program nazywa się \textit{Hello, world!} i jego jedynym zadaniem
jest wypisanie na ekran tego właśnie tekstu. Używa się go, między innymi,
do testowania konfiguracji środowiska -- ze względu na jego prostotę bardzo
ciężko jest popełnić w nim błąd, stąd jakiekolwiek anomalie wynikać muszą
z błędnej konfiguracji.

Będzie to też najprawdopodobniej pierwszy program, jaki napiszesz w każdym 
języku programowania, jakiego się kiedykolwiek nauczysz. 

Ze względu na specyfikę języka C++, zrozumienie za co odpowiada każdy fragment 
powyższego kodu może być dla Ciebie w tym momencie trudne. Dlatego nie będziemy
się tym teraz zajmować. Możemy jednak spróbować nieco zmienić powyższy kod.

\begin{try}
Zmodyfikuj program z listingu~\ref{lst:hello} tak, aby wypisywał na ekran tekst 
,,My second program''.
\end{try}

Jak pewnie się domyślasz, za właściwe wyświetlenie tekstu odpowiada linia
kodu zaczynająca się od \texttt{cout}. Możemy zmienić zawartość cudzysłowów
na inny tekst, który zostanie wyświetlony w zamian.

\begin{try}
Zmodyfikuj program z listingu~\ref{lst:hello} tak, aby wypisywał na ekran tekst
,,Hello, world!'' dwukrotnie, w dwóch osobnych liniach.
\end{try}

Kod programu wykonywany jest z reguły linijka po linijce. Dlatego żeby
wyświetlić ,,Hello, world!'' dwukrotnie, wystarczy skopiować i wkleić jeszcze
raz poniżej linię zaczynającą się od \texttt{cout}.

\begin{try}
Zmodyfikuj program z poprzedniego zadania tak, aby wypisywał w pierwszej linii
tekst ,,Hello, world!'', a w drugiej ,,Hello, again!''.
\end{try}

Ponownie, wystarczy zmienić zawartość cudzysłowów w drugiej z linii kodu
wyświetlających tekst.

\section{Wypisywanie danych}
\section{Przeprowadzanie prostych obliczeń}

\chapter{Zmienne i stałe}
\section{Tworzenie zmiennych}
\section{Modyfikowanie wartości zmiennych}
\section{Podstawowe typy danych}
\section{Wykorzystywanie zmiennych w obliczeniach}
\section{Pobieranie danych od użytkownika}

\chapter{Wyrażenia warunkowe}
\section{Porównywanie wartości}
\section{Wyrażenie \texttt{if}}
\section{Wyrażenie \texttt{else}}

\chapter{Pętle}
\section{Pętla \texttt{while}}
Listing~\ref{lst:hello}
\section{Pętla \texttt{for}}

\chapter{Tablice}

\chapter{Funkcje}
\section{Abstrakcja}
\section{Duplikacja kodu}
\section{Tworzenie funkcji bez argumentów i wartości zwracanej}
\section{Funkcje przyjmujące argumenty}
\section{Zwracanie wartości}
\section{Rekurencja}


\part{Podstawy programowania obiektowego}
\chapter{Wprowadzenie}
\section{Korzystanie ze środowiska IntelliJ IDEA}
\section{Pierwszy program w języku Java}
\section{Podobieństwa i różnice między językami Java i C++}

\chapter{Obiekty i klasy}
\section{Abstrakcja: przypomnienie}
\section{Obiekt jako przykład abstrakcji}
\section{Klasa jako szablon dla obiektów}
\section{Definiowanie własnych klas}

\chapter{Enkapsulacja}
\section{Czym jest enkapsulacja?}
\section{Poziom dostępu \texttt{public}}
\section{Poziom dostępu \texttt{private}}

\chapter{Dziedziczenie i polimorfizm}
\section{Duplikacja kodu: przypomnienie}
\section{Dziedziczenie}
\section{Dziedziczenie w języku Java}
\section{Polimorfizm}

\chapter{* Interfejsy}
\section{Klasa abstrakcyjna}
\section{Wielokrotne dziedziczenie}
\section{Interfejsy}
\section{Biblioteka standardowa Javy}

\end{document}
